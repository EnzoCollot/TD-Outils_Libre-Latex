\documentclass[12pt]{article} 

\usepackage[utf8]{inputenc}
\usepackage[OT1]{fontenc}

% Gestion de la langue du document
\usepackage[french]{babel}

% Gestion des espaces
\usepackage{xspace}

% Gestion des images
\usepackage{graphicx}

% Pour permettre la redéfinition des dimensions
\usepackage[a4paper]{geometry}

%Gestion des images dans le document
\usepackage{subcaption}

%Gestion des lien hypertext
\usepackage{hyperref}

% Configuration des dimensions
\geometry{%
left=20mm,width=165mm,
top=15mm,height=267mm,
footskip=15mm}

% Le titre
\title{Rapport_TD_Outils-Libres}
\author{Enzo Collot}
\date{Année universitaire 2022-2023}


\begin{document}

    \thispagestyle{empty}
    \begin{center}
        \includegraphics[width=12cm]{logo_iut.jpg}
        \end{center}

\vspace{1cm}

\noindent
{\large
  IUT Nancy Charlemagne\\
  Université de Lorraine\\
  2 ter boulevard Charlemagne\\
  BP 55227\\
  54052 Nancy Cedex\\[5mm]
  Département informatique
}

\vspace{5cm}

\begin{center}
    {\huge
      \textbf{Rapport TD Outils-Libres en Latex}
    }
\end{center}

\vspace{5cm}

% \vspace{2cm}
\vfill


{\Large
  \noindent
  Etudiant : Enzo Collot\\
  Année universitaire 2022--2023
}

% Ajout d'une page vide
\newpage
\thispagestyle{empty}
\mbox{}
\newpage

\newpage
% Table des matières
\tableofcontents

\newpage

\section{Efficacité de l'environnement de travail}

  \subsection{TD-1 : La souris}

\begin{itemize}
  \item Désactiver votre souris au niveau système avec la commande xinput
\end{itemize}

\vspace{0.3cm}

Pour désactiver la souris au niveau système, nous allons utiliser la commande \texttt{xinput}.

\vspace{0.3cm}

Je tiens à préciser que je vais effectuer ce test sur mon ordinateur personnel qui fonctionne sous Linux Mint.

\vspace{0.3cm}

Voici ci-dessous, la commande qui permet de désactiver la souris :

\begin{verbatim}
xinput set-prop "device name" "Device Enabled" 0
\end{verbatim}

Dans mon cas, j'ai dû rentrer la commande suivante :

\begin{verbatim}
xinput set-prop "Logitech Wireless Receiver Mouse" "Device Enabled" 1
\end{verbatim}

\vspace{0.3cm}

\begin{itemize}
  \item Initialiser un fichier dans lequel nous allons lister tous les problèmes d'efficacité rencontrés pendant cette séance.
\end{itemize}
\vspace{0.3cm}

\begin{tabular}{|c|p{5cm}|p{10cm}|}
  \hline
  \textbf{Priorité} & \textbf{Problème} & \textbf{Correctif}\\
  \hline 
  1 & Logout difficile au clavier & Raccourci clavier \textbf{Ctrl+Alt+Suppr/Delete}\\
  \hline
  2 & Impossible d'éditer des documents PDF avec Google Drive & Utilisation de LaTeX\\
  \hline
  3 & La souris est bloquée et ne répond plus & Utilisez le raccourci clavier \textbf{Ctrl+Alt+F1} pour ouvrir une console en mode texte, puis connectez-vous à votre session. Ensuite, utilisez la commande \textbf{sudo service gdm3 restart} pour redémarrer le serveur d'affichage et réinitialiser la souris \\
  \hline
  4 & Impossible d'avoir plusieurs terminaux en parallèle & Sous Terminator, \textbf{Ctrl+Shift+O} pour split le terminal verticalement,\newline \textbf{Ctrl+Shift+L} pour split le terminal horizontalement\\ 
  \hline
  5 & Accèder au navigateur de fichier & shotcut configurable dans les settings ex : \textbf{Alt+F}\\
  \hline
  6 & Naviguer dans discord sans la souris & \textbf{Tab} pour se déplacer de haut en bas, \textbf{Shift+Tab} pour aller de bas en haut \newline et \textbf{Ctrl+Tab} pour naviguer de gauche à droite\\
  \hline
  7 & La souris ne fonctionne pas sur un ordinateur portable & Utilisez le raccourci clavier \textbf{Fn+F9} pour activer ou désactiver le pavé tactile, qui peut parfois interférer avec la souris \\
  \hline

\end{tabular}

\newpage

  \subsection{TD-2 : Le clavier}

\begin{itemize}
  \item Identifier un site permettant de s'améliorer à taper au clavier.
\end{itemize}

\vspace{0.3cm}

J'ai trouvé un site qui permet de tester la rapidité de frappe au clavier. Voici le lien ci-dessous  : 

\href{https://www.taptouche.com/fr/test-de-vitesse}{Site de TapTouch}

\vspace{0.3cm}

Pourquoi celui-ci plutôt qu'un autre ?

\vspace{0.3cm}

J'ai choisie ce site car il explique quelques informations sur notre score à la fin et il nous donne des astuces pour nous améliorer.

\vspace{0.3cm}

Inclure quelques screenshots montrant l'interface

\vspace{0.3cm}

Screen de l'interface du site :

\vspace{0.3cm}

\begin{figure}[h]
  \centering
  \begin{subfigure}{0.45\textwidth}
    \centering
    \includegraphics[width=\textwidth]{image_site_tap_touche.png}
    \caption{Page principale du site}
  \end{subfigure}
  \vspace{0.9cm} % Espace verticale entre les images
  \begin{subfigure}{0.45\textwidth}
    \centering
    \includegraphics[width=\textwidth]{image_site_tap_touche2.png}
    \caption{Page pour effectuer le test de rapidité}
  \end{subfigure}
  \caption{Deux captures d'écran du site TapTouch}
\end{figure}

\vspace{0.3cm}

Essayer de masquer ses mains pour taper sans regarder

\vspace{0.3cm}

Inclure quelques résultats de rapidité.

\vspace{0.3cm}

Premier test effectué le 22/12/2022 : 

\begin{center}
  \includegraphics[width=10cm]{Premier_test.png}
\end{center}

\vspace{0.3cm}

\newpage

Deuxième test effectué le 23/12/2022 :

\begin{center}
  \includegraphics[width=10cm]{Deuxième_test.png}
\end{center}
